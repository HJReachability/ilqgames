 

Iterative Linear-\/\+Quadratic Games -\/ a new, real-\/time solver for {\itshape N}-\/player general differential games.

{\itshape N\+O\+TE} that this project is still under active development and is only intended to be used as an academic reference at this stage. However, my intent is that it should eventually become stable enough for use as a reliable substitute for a model-\/predictive control backend.

\subsection*{Paper}

For a full description of the algorithm itself and examples of how it can be applied in collision-\/avoidance problems, please refer to the \href{https://arxiv.org/abs/1909.04694}{\tt paper}, which was presented at I\+C\+RA 2020.

If you find this repository useful, please do cite the paper\+: 
\begin{DoxyCode}
1 @inproceedings\{fridovich2020efficient,
2   title=\{Efficient iterative linear-quadratic approximations for nonlinear multi-player general-sum
       differential games\},
3   author=\{Fridovich-Keil, David and Ratner, Ellis and Peters, Lasse and Dragan, Anca D and Tomlin, Claire
       J\},
4   booktitle=\{2020 IEEE International Conference on Robotics and Automation (ICRA)\},
5   pages=\{1475--1481\},
6   year=\{2020\},
7   organization=\{IEEE\}
8 \}
\end{DoxyCode}
 If you are interested in the feedback linearized version of this algorithm, please also refer to that \href{https://arxiv.org/abs/1910.00681}{\tt paper}, which is also published at I\+C\+RA 2020\+: 
\begin{DoxyCode}
1 @inproceedings\{fridovich2020iterative,
2   title=\{An iterative quadratic method for general-sum differential games with feedback linearizable
       dynamics\},
3   author=\{Fridovich-Keil, David and Rubies-Royo, Vicenc and Tomlin, Claire J\},
4   booktitle=\{2020 IEEE International Conference on Robotics and Automation (ICRA)\},
5   pages=\{2216--2222\},
6   year=\{2020\},
7   organization=\{IEEE\}
8 \}
\end{DoxyCode}


\subsection*{Primary contributor}

The primary contributor is \href{https://dfridovi.github.io}{\tt David Fridovich-\/\+Keil}, a recent PhD grad from Claire Tomlin\textquotesingle{}s group in the E\+E\+CS department at UC Berkeley. David is currently a postdoc at Stanford with Mac Schwager, and will start as a professor at UT Austin in fall 2021. The best way to contact David is by email, {\itshape david.\+fridovichkeil at stanford dot edu}, or if you have specific questions about this repository, please post an \href{https://github.com/HJReachability/ilqgames/issues}{\tt issue}.

\subsection*{Documentation}

Documentation is automatically generated by a continuous integration service with each push to the {\ttfamily master} branch, and it may be found \href{https://HJReachability.github.io/ilqgames/documentation/html/}{\tt here}. Additionally, a more high-\/level form of documentation is available in P\+DF form \href{https://github.com/HJReachability/ilqgames/blob/master/ILQGames_Documentation.pdf}{\tt here}.

\subsection*{Language}

An early version of the iterative linear-\/quadratic game algorithm was implemented in Python; it is now deprecated and stored in the {\ttfamily python/} directory. Ongoing and future development will proceed primarily in C++, although a colleague has built a standalone version in Julia \href{https://github.com/lassepe/iLQGames.jl}{\tt here}

\subsection*{Dependencies}

{\itshape ilqgames} has been tested both in OS X and Ubuntu. Depending on your version of Ubuntu, you may notice a linker error related to {\ttfamily aclocal-\/\+XX}, which may be fixed by symlinking Ubuntu\textquotesingle{}s native {\ttfamily aclocal} to the desired executable {\ttfamily aclocal-\/\+XX}. Otherwise, external dependencies are standard\+:


\begin{DoxyItemize}
\item {\ttfamily glog} (Google\textquotesingle{}s logging tools)
\item {\ttfamily gflags} (Google\textquotesingle{}s command line flag tools)
\item {\ttfamily opengl}, {\ttfamily glut} (graphics tools)
\item {\ttfamily eigen3} (linear algebra library)
\end{DoxyItemize}

\subsection*{Getting started}

This repository uses the {\ttfamily cmake} build system and may be compiled following standard cmake protocols\+: 
\begin{DoxyCode}
1 mkdir build && cd build
2 cmake ..
3 make -j8
\end{DoxyCode}


Executables will be stored in the {\ttfamily bin/} directory, with the exception of {\ttfamily build/run\+\_\+tests}, which runs unit tests. To run unit tests from the build directory, simply execute\+: 
\begin{DoxyCode}
1 ./run\_tests
\end{DoxyCode}


To run a particular example from the {\ttfamily bin/} directory, e.\+g., the three-\/player intersection\+: 
\begin{DoxyCode}
1 ./three\_player\_intersection
\end{DoxyCode}


With any executable, a full explanation of command line arguments can be found by running\+: 
\begin{DoxyCode}
1 ./<name-of-executable> --help
\end{DoxyCode}


\subsection*{Extending {\itshape ilqgames}}

\subsubsection*{New examples}

All specific examples, e.\+g. the three-\/player intersection, inherit from the base class {\ttfamily Problem}. This base class provides a number of utilities, such as warm starting, and also wraps the solver. To inherit, simply write a constructor which instantiates the appropriate solver with costs, and also set the initial state, operating point, and strategies. To see an example, consult the \href{https://github.com/HJReachability/ilqgames/blob/master/src/three_player_intersection_example.cpp}{\tt Three\+Player\+Intersection\+Example} class.

\subsubsection*{R\+OS}

There is a companion R\+OS extension under development, which may be found \href{https://github.com/HJReachability/ilqgames_ros}{\tt here}. It is currently still marked private, but rest assured that the usage of the {\itshape ilqgames} toolbox in R\+OS is straightforward! If you want to implement this yourself for a custom application it is not too hard.

\subsubsection*{G\+UI}

The native {\itshape ilqgames} graphical user interface can be a very useful tool in debugging algorithm changes, new costs, etc. Currently, it supports so-\/called {\ttfamily Top\+Down\+Renderable\+Problem} games, where the state encodes x-\/y position and (potentially) heading information. To extend the G\+UI to other types of games, simply write a custom renderer, e.\+g. mirroring the {\ttfamily Top\+Down\+Renderer} class. All G\+UI functionality is based on the \href{https://github.com/ocornut/imgui}{\tt Dear\+Im\+Gui} library, a copy of which is maintained locally in the {\ttfamily external/} directory. 