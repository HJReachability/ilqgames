\href{https://travis-ci.org/HJReachability/ilqgames}{\tt } \mbox{[}\mbox{]}(L\+I\+C\+E\+N\+SE)

Iterative Linear-\/\+Quadratic Games -\/ a new, real-\/time solver for {\itshape N}-\/player general differential games.

\subsection*{Paper}

For a full description of the algorithm itself and examples of how it can be applied in collision-\/avoidance problems, please refer to the \href{https://arxiv.org/abs/1909.04694}{\tt paper}.

If you find this repository useful, please do cite the paper\+: 
\begin{DoxyCode}
1 @misc\{fridovichkeil2019efficient,
2     title=\{Efficient Iterative Linear-Quadratic Approximations for Nonlinear Multi-Player General-Sum
       Differential Games\},
3     author=\{David Fridovich-Keil and Ellis Ratner and Anca D. Dragan and Claire J. Tomlin\},
4     year=\{2019\},
5     eprint=\{1909.04694\},
6     archivePrefix=\{arXiv\},
7     primaryClass=\{eess.SY\}
8 \}
\end{DoxyCode}


Currently, the paper is under review at I\+E\+EE Robotics and Automation Letters. An updated link and reference will be posted upon publication.

\subsection*{Primary contributor}

The primary contributor is \href{https://people.eecs.berkeley.edu/~dfk/}{\tt David Fridovich-\/\+Keil}, a fifth-\/year PhD student advised by Claire Tomlin in the E\+E\+CS department at UC Berkeley. The best way to contact David is by email, {\itshape dfk at eecs dot berkeley dot edu}, or if you have specific questions about this repository, please post an \href{https://github.com/HJReachability/ilqgames/issues}{\tt issue}.

\subsection*{Language}

An early version of the iterative linear-\/quadratic game algorithm was implemented in Python; it is now deprecated and stored in the {\ttfamily python/} directory. Ongoing and future development will proceed primarily in C++, although a colleague is building a version in Julia, which may eventually be merged here.

\subsection*{Dependencies}

{\ttfamily ilqgames} has been tested both in OS X and Ubuntu. Depending on your version of Ubuntu, you may notice a linker error related to {\ttfamily aclocal-\/\+XX}, which may be fixed by symlinking Ubuntu\textquotesingle{}s native {\ttfamily aclocal} to the desired executable {\ttfamily aclocal-\/\+XX}. Otherwise, external dependencies are standard\+:


\begin{DoxyItemize}
\item {\ttfamily glog} (Google\textquotesingle{}s logging tools)
\item {\ttfamily gflags} (Google\textquotesingle{}s command line flag tools)
\item {\ttfamily opengl}, {\ttfamily glut} (graphics tools)
\item {\ttfamily eigen3} (linear algebra library)
\end{DoxyItemize}

\subsection*{Getting started}

This repository uses the {\ttfamily cmake} build system and may be compiled following standard cmake protocols\+: 
\begin{DoxyCode}
1 mkdir build && cd build
2 cmake ..
3 make -j8
\end{DoxyCode}


Executables will be stored in the {\ttfamily bin/} directory, with the exception of {\ttfamily build/run\+\_\+tests}, which runs unit tests. To run unit tests from the build directory, simply execute\+: 
\begin{DoxyCode}
1 ./run\_tests
\end{DoxyCode}


To run a particular example from the {\ttfamily bin/} directory, e.\+g., the three-\/player intersection\+: 
\begin{DoxyCode}
1 ./three\_player\_intersection
\end{DoxyCode}


With any executable, a full explanation of command line arguments can be found by running\+: 
\begin{DoxyCode}
1 ./<name-of-executable> --help
\end{DoxyCode}


\subsection*{Documentation}

Documentation is automatically generated by a continuous integration service with each push to the {\ttfamily master} branch, and it may be found \href{https://HJReachability.github.io/ilqgames/documentation/html/}{\tt here}.

\subsection*{Extending {\itshape ilqgames}}

\subsubsection*{New examples}

All specific examples, e.\+g. the three-\/player intersection, inherit from the base class {\ttfamily Problem}. This base class provides a number of utilities, such as warm starting, and also wraps the solver. To inherit, simply write a constructor which instantiates the appropriate solver with costs, and also set the initial state, operating point, and strategies. To see an example, consult the \href{https://github.com/HJReachability/ilqgames/blob/master/src/three_player_intersection_example.cpp}{\tt Three\+Player\+Intersection\+Example} class.

\subsection*{R\+OS}

There is a companion R\+OS extension under development, which may be found \href{https://github.com/HJReachability/ilqgames_ros}{\tt here}. It is currently still marked private, but rest assured that the usage of the {\itshape ilqgames} toolbox in R\+OS is straightforward! If you want to implement this yourself for a custom application it is not too hard.

\subsection*{G\+UI}

The native {\itshape ilqgames} graphical user interface can be a very useful tool in debugging algorithm changes, new costs, etc. Currently, it supports so-\/called {\ttfamily Top\+Down\+Renderable\+Problem} games, where the state encodes x-\/y position and (potentially) heading information. To extend the G\+UI to other types of games, simply write a custom renderer, e.\+g. mirroring the {\ttfamily Top\+Down\+Renderer} class. All G\+UI functionality is based on the \href{https://github.com/ocornut/imgui}{\tt Dear\+Im\+Gui} library, a copy of which is maintained locally in the {\ttfamily external/} directory. 